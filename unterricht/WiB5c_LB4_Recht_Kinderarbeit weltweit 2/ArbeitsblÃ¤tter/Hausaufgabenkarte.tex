\documentclass[12pt, a4paper]{article}
\usepackage[top=2cm, bottom=2cm, left=2cm, right=2cm]{geometry}
\usepackage[utf8]{inputenc}
\usepackage[T1]{fontenc}
\usepackage{array}
\usepackage{xcolor}
\usepackage{enumitem}
\usepackage[ngerman]{babel}
\usepackage{framed}
\usepackage{tgbonum}

% Define colors
\definecolor{headercolor}{RGB}{231, 76, 60}
\definecolor{boxcolor}{RGB}{249, 235, 234}

% Define environments
\newenvironment{taskbox}{%
    \begin{framed}
        \color{black}
    }{%
    \end{framed}
}

\setlength{\parindent}{0pt}

\begin{document}
    \begin{center}
        \vspace{0.3cm}
        {\color{headercolor}\Huge\textbf{HAUSAUFGABE: REGIERUNGSPLANSPIEL}}\\[0.5cm]
        \textbf{Name: \rule{4cm}{0.4pt} \hspace{1cm} Datum: \rule{3cm}{0.4pt}}
    \end{center}

    \vspace{0.3cm}

    \begin{taskbox}
        \Large\textbf{Du bist jetzt Teil der Regierung eines neuen Landes!}\\
        \vspace{0.5cm}
        \textbf{Deine Aufgabe:}\\[0.3cm]
        \large\textbf{ Entwickle ein vollständiges Regelwerk zum Schutz von Kindern bei der Arbeit.}
    \end{taskbox}

    \vspace{0.5cm}

    \textbf{Dein Regelwerk sollte enthalten:}

    \vspace{0.3cm}

    \begin{taskbox}
        \textbf{1. Ab welchem Alter dürfen Kinder arbeiten?}\\
        \rule{\linewidth}{0.4pt}

        \vspace{0.5cm}
        \textbf{2. Wie viele Stunden dürfen Kinder maximal arbeiten?}\\
        \rule{\linewidth}{0.4pt}

        \vspace{0.5cm}
        \textbf{3. Welche Arbeiten sind für Kinder verboten?}\\
        \rule{\linewidth}{0.4pt}\\
        \rule{\linewidth}{0.4pt}

        \vspace{0.5cm}
        \textbf{4. Wie wird sichergestellt, dass Kinder zur Schule gehen können?}\\
        \rule{\linewidth}{0.4pt}\\
        \rule{\linewidth}{0.4pt}

        \vspace{0.5cm}
        \textbf{5. Was passiert, wenn jemand die Regeln nicht einhält?}\\
        \rule{\linewidth}{0.4pt}\\
        \rule{\linewidth}{0.4pt}
    \end{taskbox}

    \vspace{0.5cm}

    \colorbox{boxcolor}{%
        \begin{minipage}{\dimexpr\linewidth-2\fboxsep\relax}
            \textit{Gestalte dein Regelwerk im Heft. Du kannst es auch als Plakat gestalten oder mit Bildern und Beispielen ergänzen. Sei kreativ!}
        \end{minipage}%
    }

    \vspace{0.5cm}
    \begin{flushright}
        \textbf{Abgabe: \rule{4cm}{0.4pt}}
    \end{flushright}
\end{document}
